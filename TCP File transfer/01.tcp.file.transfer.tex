\documentclass[a4paper,12pt]{article}

\usepackage{graphicx}
\usepackage{listings}
\usepackage{courier}
\usepackage{hyperref}

\lstset{
    basicstyle=\ttfamily\small,
    breaklines=true
}

\title{Practical Work 1: TCP File Transfer}
\author{Name: Bui Truong An - 22BA13001}
\date{}

\begin{document}

\maketitle

\section{Goal}
The objective of this practical work is to implement a one-to-one file transfer over TCP/IP using a command-line interface. The system is based on the provided chat architecture and must include:
\begin{itemize}
    \item One server
    \item One client
    \item Communication through TCP sockets
\end{itemize}

\section{Protocol Design}
We designed a simple application-level protocol for file transfer:
\begin{itemize}
    \item The client sends a \texttt{REQUEST <filename>} command to the server.
    \item The server checks if the file exists.
    \item If the file exists, the server responds with \texttt{OK <filesize>} and starts sending raw bytes.
    \item If the file does not exist, the server responds \texttt{ERROR}.
\end{itemize}

\subsection*{Protocol Diagram}

\begin{verbatim}
Client                             Server
  |                                  |
  |--- REQUEST filename -----------> |
  |                                  |
  | <----- OK filesize ------------- |
  |                                  |
  | <----- file data bytes --------- |
  |                                  |
\end{verbatim}

\section{System Organization}

\subsection*{Architecture Overview}
The system consists of two standalone programs:

\begin{itemize}
    \item \textbf{Server}: waits for TCP connections and sends files.
    \item \textbf{Client}: connects to the server and receives files.
\end{itemize}

\subsection*{System Diagram}

\begin{verbatim}
+-----------+        TCP        +-----------+
|  Client   | <----------------> |  Server   |
+-----------+                    +-----------+
       |                             |
       | requests file               | reads and sends file
       | saves received file         | handles error conditions
\end{verbatim}

\section{Implementation}

Below are simplified code snippets showing the implementation idea.

\subsection{Server Code Snippet }
\begin{lstlisting}[language=Python]
import socket, os

s = socket.socket()
s.bind(("0.0.0.0", 9000))
s.listen(1)

conn, addr = s.accept()
request = conn.recv(1024).decode()

cmd, filename = request.split()

if cmd == "REQUEST" and os.path.isfile(filename):
    size = os.path.getsize(filename)
    conn.send(f"OK {size}".encode())
    
    with open(filename, "rb") as f:
        conn.sendall(f.read())
else:
    conn.send(b"ERROR")

conn.close()
s.close()
\end{lstlisting}

\subsection{Client Code Snippet }
\begin{lstlisting}[language=Python]
import socket

filename = "example.txt"
s = socket.socket()
s.connect(("127.0.0.1", 9000))

s.send(f"REQUEST {filename}".encode())
response = s.recv(1024).decode()

status, value = response.split()

if status == "OK":
    size = int(value)
    data = b""
    while len(data) < size:
        data += s.recv(4096)

    with open("received_" + filename, "wb") as f:
        f.write(data)

s.close()
\end{lstlisting}

\end{document}
